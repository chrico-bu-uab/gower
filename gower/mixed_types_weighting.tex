%! Author = temp
%! Date = 3/24/23

% Preamble
\documentclass[11pt]{article}

% Packages
\usepackage{amsmath}
\DeclareMathOperator{\Var}{\widehat{Var}}
\DeclareMathOperator{\Kurt}{\widehat{Kurt}}
\usepackage[legalpaper, landscape, margin=2in]{geometry}
\usepackage{algpseudocode}

% Document
\begin{document}

    The following is a scheme for weighting variables in a mixed-type dataset to convert into a Gower
    matrix.

    $x_t$ is the $t^{\text{th}}$ variable of dataset $X$.

    If $x_t$ is a numeric variable, then $w_t\leftarrow1$.

    Otherwise, let ${h_t}_k$ be the $k^{\text{th}}$ column of the one-hot encoding $H_t$ of $x_t$.

    $w_t\leftarrow\sum_k^{K_t}{\Var({h_t}_k)}\left[1-\sum_k^{K_t}{\Var({h_t}_k})\right]\Kurt(H_t)$

    The first term is zero when all the values of the variable are the same, whereas the second term is zero when
    they're all different. The third term increases as the number of unique values increases.

\end{document}
