%! Author = temp
%! Date = 3/24/23

% Preamble
\documentclass[11pt]{article}

% Packages
\usepackage{amsmath}
\DeclareMathOperator{\Var}{\widehat{Var}}
\DeclareMathOperator{\Kurt}{\widehat{Kurtosis}}
\usepackage[legalpaper, landscape, margin=2in]{geometry}
\usepackage{algpseudocode}

% Document
\begin{document}

    The following is a scheme for weighting variables in a mixed-type dataset to convert into a Gower
    matrix.

    $x$ is a column of matrix $X$ of shape $\left(n,d\right)$.

    If $x$ is a numeric variable, then $w\leftarrow1$.

    Otherwise, if $x$ is a categorical variable, then we can use the following equation to calculate the entry-wise
    unbiased sample kurtosis of $H$, the one-hot encoding of $x$, where $h_c$ is the column of $H$ corresponding to the
    $c^{\text{th}}$ of $C$ classes in $x$:

    \begin{equation}
        \Kurt=\frac{(C^2n^2-1)(C+\frac{1}{(C-1)}-2)-3(Cn-1)^2}{(Cn-2)(Cn-3)}+3 \label{eq:kurtosis}
    \end{equation}

    Then, we can calculate the weight of $x$ as follows:

    \begin{equation}
        w\leftarrow
        \left[\sum_c^C{\Var(h_c)}\right]
        \left[1-\sum_c^C{\Var(h_c)}\right]
        \Biggl[\Kurt\Biggr]
        \label{eq:equation}
    \end{equation}



    The first term in brackets is zero when all the values of the variable are the same, the second term is zero when
    they're all different, and the third term increases with the number of classes and decreases with the number of
    observations in each class.

\end{document}
